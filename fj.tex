\documentclass[12pt]{article}
\usepackage{bbold}
\usepackage{semantic}
\usepackage{mathtools}
\usepackage{syntax}
\usepackage{xstring}
\usepackage{trfrac}
\usepackage{listings}
\usepackage{color}



\begin{document}
\lstdefinelanguage{diff}{
  morecomment=[f][\color{blue}]{@@},     % group identifier
  morecomment=[f][\color{red}]-,         % deleted lines
  morecomment=[f][\color{green}]+,       % added lines
  morecomment=[f][\color{magenta}]{---}, % Diff header lines (must appear after +,-)
  morecomment=[f][\color{magenta}]{+++},
}

\lstset{
  language=Java,
  basicstyle=\ttfamily
}

\section{Easy Examples}

\begin{enumerate}
  \item ``Alpha equivalence'', change from \texttt{bak} to \texttt{bing}

    \lstinputlisting[language=diff]{code/alpha.diff}

  \item External equivalence, body of \texttt{baz} adds 10

    \lstinputlisting[language=diff]{code/external.diff}

  \item Class body changes,

    \lstinputlisting[language=diff]{code/subclass.diff}

\newpage

\section{Other Examples}

\begin{enumerate}
  \item Class Reference addition, \texttt{Qux} instance returned. To avoid recompilation \texttt{Qux} would have to type check. That is it would have to exist and be a subtype of \texttt{int} in the old class table. If \texttt{Qux} does not yet exist then local only modifications aren't possible, because \texttt{Qux} has to be compiled before we can typecheck.

    \lstinputlisting[language=diff]{code/new-ref.diff}
\end{enumerate}

\section{Eval Derivations to Eval Derivations}

\begin{enumerate}
  \item assumes type c
\end{enumerate}

$\mathtt{(Pair)(new\ Pair(new\ Pair(new\ A(),\ new\ B()),\ new\ B()).first)}$
\vspace{-0.1cm}
\begin{equation*}
  \trfrac[E-PROJNEW]{
    \trfrac[E-CASTNEW]{
      \mathtt{(new\ Pair(new\ A(),\ new\ B()))}
    }{
      \mathtt{(Pair)(new\ Pair(new\ A(),\ new\ B()))}
    }
  }{
    \mathtt{(Pair)(new\ Pair(new\ Pair(new\ A(),\ new\ B()),\ new\ B()).first)}
  }
\end{equation*}

\vspace{0.4cm}

$\mathtt{(Pair)(new\ Pair(new\ Pair(new\ C(),\ new\ B()),\ new\ B()).first)}$
\vspace{-0.1cm}
\begin{equation*}
  \trfrac[E-PROJNEW]{
    \trfrac[E-CASTNEW]{
      \mathtt{(new\ Pair(new\ C(),\ new\ B()))}
    }{
      \mathtt{(Pair)(new\ Pair(new\ C(),\ new\ B()))}
    }
  }{
    \mathtt{(Pair)(new\ Pair(new\ Pair(new\ C(),\ new\ B()),\ new\ B()).first)}
  }
\end{equation*}

\vspace{0.4cm}

$\mathtt{(Pair)(new\ Pair(new\ Pair(new\ A(),\ new\ B()),\ new\ B()))}$
\vspace{-0.1cm}
\begin{equation*}
  \trfrac[E-CASTNEW]{
    \mathtt{new\ Pair(new\ Pair(new\ A(),\ new\ B()),\ new\ B())}
  }{
    \mathtt{(Pair)(new\ Pair(new\ Pair(new\ A(),\ new\ B()),\ new\ B()))}
  }
\end{equation*}


\section{Notes}

\begin{enumerate}
  \item Where is the line between acceptable changes and unacceptable? How does it relate to the semantics/type rules of Featherweight Java?
  \item Consider the types and semantics separately

\end{enumerate}

\section{Ideas}

\begin{enumerate}
  \item Direct alterations to optimized binary class file to avoid re-JIT
\end{enumerate}

\section{Issues}

\begin{enumerate}
  \item Dynamically updating optimized state might be impossible unless transformations can be composed with the differential update
\end{enumerate}



\end{document}
