\documentclass[12pt]{article}
\usepackage{bbold}
\usepackage{semantic}
\usepackage{mathtools}
\usepackage{syntax}
\usepackage{xstring}
\usepackage{trfrac}

\renewcommand{\syntleft}{\normalfont\itshape}
\renewcommand{\syntright}{}

\newcommand{\cmp}[2]{
  \mathcal{#1}|[ \mathtt{ \StrSubstitute{#2}{ }{\ } } |]
}

\newcommand{\syn}[1]{
  \synt{\textbf{#1}}
}

\begin{document}
\section{Type Rules}

\begin{equation*}
  \trfrac[Relation]{
  }{
    \Gamma, R.C, R.a_1:\tau_1,\ldots,R.a_n:\tau_n  \vdash R : \{a_1:\tau_1,\ldots,a_n:\tau_n\}\ |\ C
  }
\end{equation*}

\begin{equation*}
  \trfrac[\ Select]{
    \Gamma \vdash R : \vec{\tau_1}\ |\ C
  }{
    \Gamma \vdash \sigma_{\varphi}(R) : \vec{\tau_1}\ |\ C \cup \{\varphi\}
  }
\end{equation*}

\begin{equation*}
  \trfrac[\ Subtype*]{
    \Gamma \vdash R : \vec{\tau_1}\ |\ C_1
    \qquad \Gamma \vdash \pi_{a_1,\ldots,a_m}(R) : \vec{\tau_2}\ |\ C_2
  }{
    \Gamma \vdash \vec{\tau_1} \le \vec{\tau_2}\ |\ C_1 \cup C_2
  }
\end{equation*}

\begin{equation*}
  \trfrac[\ Project*]{
    \Gamma \vdash R : \vec{\tau_1}\ |\ C
    \qquad \vec{\tau_2} = \{ a_1:\tau_1,...,a_m:\tau_m\}
    \qquad \vec{\tau_2} / \vec{\tau_1} = \emptyset
  }{
    \pi_{a_1,\ldots,a_m}(R) : \vec{\tau_2} \cap \vec{\tau_1}\ |\ C
  }
\end{equation*}


\begin{equation*}
  \trfrac[\ Join*]{
    \Gamma \vdash R : \vec{\tau_1}\ |\ C_1
    \qquad \ S : \vec{\tau_2}\ |\ C_2
  }{
   \Gamma \vdash  R \bowtie S : \vec{\tau_1} \sqcup \vec{\tau_2}\ |\ C_1 \cup C_2
  }
\end{equation*}

\section{Notes}

\subsection{TODO}
\begin{enumerate}
  \item the Project rule the a's cannot convert types, create columns by conversions. Duped columns still a subtype?
  \item the subtype rule has to account for columns, column types, and constraints. That is, a \lit*{select} or \lit*{insert} should type check as a super type of it's target table
  \item the join rule has to account for constraints
\end{enumerate}

\subsection{Type definitions}

\begin{enumerate}
  \item columns and their types are type definitions
  \item column/table constraints are type constraints
  \item referential constraints are constraints from external sources
  \item constraints are a weak form a schema modeling
\end{enumerate}

\subsection{Type checks}

\begin{enumerate}
  \item columns references must respect type constraints
  \item where/having/order clauses should respect type/data constraints, it's alright to select and get no tuples, it's not alright to select in a way that cannot return tuples
  \item join results should respect proxied referential data constraints
  \item column type alterations must/should respect type constraints
\end{enumerate}

\subsection{Value}
\begin{enumerate}
  \item Prevent type mismatch
    \begin{enumerate}
      \item selection, \lit*{... LIKE `foo' ...} on \lit*{int} column
      \item projection, \lit*{ ... lower(column_name) ... } on \lit*{int} column
      \item alteration, \lit*{ ALTER TABLE ... TYPE numeric(10,2);} on \lit*{text} column
    \end{enumerate}
  \item Prevent uselessness
    \begin{enumerate}
      \item e.g. selection criteria fall outside table constraints, especially inserts/creates
    \end{enumerate}
\end{enumerate}

\subsection{Questions}
\begin{enumerate}
  \item how do you handle type conversions in the subtyping relation?
  \item In the Join type rule how would one union the constraints?
\end{enumerate}

\subsection{Brainstorm}
\begin{enumerate}
  \item dml insertions into table must respect data constraints (eg, not null, types)
  \item historical information constraints (eg, if a column is/has always been not null the data will always satisfy the constraint)
  \item build a set of constraints for the tables, check DML statments (data later still has value even if changing)
  \item identity dependencies
  \item index table dependencies
  \item referential constraint dependencies
  \item column constraint dependencies, e.g. \lit*{foo int check foo > bar} means foo depends on bar
  \item as column constraints change over time they might become incompatible
  \item constraint solving over time, temporal understanding of constraints
  \item certain types of constraints suggest certain types of index bitmap vs normal (eg, \lit*{foo int CHECK (price > 0 and price < 10)})
  \item constraints could allow for fast verification of data (eg, foo int CHECK (\lit*{price > 0 and price =< 10} + unique) means check the table for 10 items) on foreign key checks
  \item expansion (weakening?) of constraints is sure to be ok, strengthening may not be
  \item redundant constraints
  \item pk is unique + not null, many columns is more complex
  \item delete restrictions/cascades vs other constraints?
  \item constraint relationship with queries for redundant, unnecessary, or impossible queries
  \item \lit*{insert into select ...} and \lit*{create table as ... } will satisfy the constraints of the table being selected from, even if the table doesn't require them
  \item \lit*{insert into select ...} data set should satisfy constraints of table definition
  \item \lit*{create index ...} predicates (postgres) should satisfy constraints
  \item constraint solving can produce different answers for the existing data vs the new schema, (eg, constraint removed in current schema, still constrained on existing data, reliance on existing data must reference old constraint, reliance on new schema must reference new constraint)
  \item type mismatch in definition and use of columns
  \item type mismatch in alteration of columns
  \item type width as a subtype relationship?
  \item what is the relationship between view/create as where statement constraints and regular table constraints?
  \item are constraints other than not null / default used?
  \item can table definitions be modeled as projects/selections/renames on the relation of all possible field types? bottom?
\end{enumerate}

\newpage
\section{Recursive Query Non-Termination}

\subsection{Brainstorm}
\begin{enumerate}
  \item Showing termination or non-termination is different for recursive queries because all, or a significant portion of possible, inputs are known
  \item Constraints on the table might support a proof
  \item Triggers could prevent cycles
  \item Does certain data make the semantics of a query strongly normalizing? Can a language be semantics be affected by data? Lisp?
  \item \textbf{ONLY works for direct cycles} Assume integers for pk, fk, add them and if you get a well order on the sum and $>$ then it should terminate. Because the pk should always be unique, the only way for them to be a cycle is if the pk and fk are swapped inducing a cycle \\

    \begin{tabular}{| l | c || r |}
      \hline
      Pk & Fk & Sum \\
      \hline
      1 & _ & 1 \\
      2 & 1 & 3 \\
      3 & 1 & 4 \\
      4 & 2 & 6 \\
      \hline
    \end{tabular}

    \begin{tabular}{| l | c || r |}
      \hline
      Pk & Fk & Sum \\
      \hline
      1 & _ & 1 \\
      2 & 4 & 6 \\
      3 & 1 & 4 \\
      4 & 2 & 6 \\
      \hline
    \end{tabular} \\

  \item Uniqueness constraint on the sum of the pk and fk? And how does this map to a constrain on the program?
  \item The above suggests that tuples in a relation are the states of a recursive query?
  \item Map any pk, fk into an integer with a hashing algo
  \item reducing all cycles in all relations to this problem would provide a general solution to guaranteeing termination on a finite set of data.
  \item in the case of some convoluted query that makes use of cycles can you consider the alternate termination check as part of the primary key?
  \item what is the transition relation for a given recursive query?
  \item prove that the query is vulnerable to cycles, prove no cycles in the data
  \item Cyclic Graphs: \\
    \begin{tabular}{| l | c || r |}
      \hline
      Pk & Fk & Sum \\
      \hline
      1 & 4 & 5 \\
      2 & 1 & 3 \\
      3 & 2 & 5 \\
      4 & 3 & 7 \\
      \hline
    \end{tabular} \\

    \begin{tabular}{| l | c || r |}
      \hline
      Pk & Fk & Sum \\
      \hline
      1 & 8 & 7 \\
      2 & 1 & 3 \\
      3 & 2 & 5 \\
      4 & 3 & 7 \\
      5 & 4 & 9 \\
      6 & 5 & 11 \\
      7 & 6 & 13 \\
      8 & 3 & 11 \\
      \hline
    \end{tabular} \\
  \item Unordered (pk/fk hashing) \\ \\

    \begin{tabular}{| l | c || r |}
      \hline
      Pk & Fk & Sum \\
      \hline
      1 & 8 & 7 \\
      2 & 4 & 6 \\
      3 & 2 & 5 \\
      4 & 6 & 10 \\
      5 & _ & 5 \\
      6 & 5 & 11 \\
      7 & 6 & 13 \\
      8 & 3 & 11 \\
      \hline
    \end{tabular} \\

    \begin{tabular}{| l | c || r |}
      \hline
      Pk & Fk & Sum \\
      \hline
      5 & _ & 5 \\
      6 & 5 & 11 \\
      4 & 6 & 10 \\
      7 & 6 & 13 \\
      2 & 4 & 6 \\
      3 & 2 & 5 \\
      8 & 3 & 11 \\
      1 & 8 & 7 \\
      \hline
    \end{tabular} \\

    \item can we come up with an algo to automatically generate a ranking function for certain recursive queries? Eg, for queries that explicitly prevent cycles

    \item harder because there's nothing to guarantee an empty set in most queries like there is to guarantee that a variable is decreasing
    \item easier because the existing data is available
    \item boils down to the data set and establishing a ranking function for it?
    \item given the cycle detection example, how does that map to a ranking function?
    \item $[1,2,3 ... 4] \rightarrow [1,2,3 ... 4, 1]$ Assume that the addition of array elements in monotonic, what is the operation that will break monotonicity for a duplicate element?
    \begin{enumerate}
      \item size of the set should always increase \\
        \verb|a = {1,2,3,...,4}; b = {1,2,3,...,4,1};| \\
        \verb|count(a) =/= count(b);|
    \end{enumerate}

  \item any property from recursive query ref involved in a join/$=$ from the table should be involved in a termination condition (very generally the pk and fk)
  \item must consider \verb|union| and \verb|union all|
 \end{enumerate}

\newpage
\section{Recursive Query Liveness}

\subsection{Brainstorm}
\begin{enumerate}
  \item nothing external to the query should prevent the query from having the liveness property of returning the results. That is, only data and the query itself should prevent the liveness property.
\end{enumerate}

\end{document}
