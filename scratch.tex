\documentclass[12pt]{article}
\usepackage{bbold}
\usepackage{semantic}
\usepackage{mathtools}
\usepackage{syntax}
\usepackage{xstring}

\renewcommand{\syntleft}{\normalfont\itshape}
\renewcommand{\syntright}{}

\newcommand{\cmp}[2]{
  \mathcal{#1}|[ \mathtt{ \StrSubstitute{#2}{ }{\ } } |]
}

\newcommand{\syn}[1]{
  \synt{\textbf{#1}}
}

\begin{document}

\section{Abstract}

Roy and Haridi \cite{RoyH2004} divide declarative languages in to two major groups based on expressive power: descriptive and programmable. In particular, descriptive declarative languages are characterized by the their limited expressive power. Indeed, languages that might otherwise be purely descriptive (e.g. SQL DDL) must be augmented with programmable declarative elements to describe anything beyond the initial system state.

We believe that the purely descriptive form is sufficiently expressive in many cases given additional information about syntactic change that is readily available in modern programming environments. To leverage that information, we develop an extension to traditional denotational semantics that supports meaning for changes in syntax. With that extension we construct a denotational semantics for standardized SQL DDL and show that it is possible to eliminate several of its non-descriptive features. Finally, we construct an adapter for an existing implementation.

\begin{description}
  \item[eliminate unnecessary features] Show equivalence?
\end{description}

\section{Candidate Theorems}
\label{sec:theorems}

Order of preference:

\begin{description}
  \item[Syntax] A well-defined descriptive declarative syntax can always be mapped directly to the desired system state.

    This follows the same line of reasoning as the other syntax theorem: i.e., the declaration should dictate system state. The wording is focused on allowing snippets of syntax within a possibly larger grammar to satisfy the theorem.

    This sounds bi-directional, but that depends on the system. The syntax should determine the system state at interpretation time. It might be that other processes alter the state after that.

    Are things like the implementation of finding the difference between versions of the syntax unimportant for the theorem?

    \vspace{0.5cm}

  \item[Syntax] A well-defined descriptive declarative semantics means there is always a clear mapping between syntax and system state.

    Note that \verb|alter| or \verb|drop| statements followed by more of the same for the same target object and properties will violate this theorem.

    This seems strict. Partial definitions may be more difficult to implement or prove but can still provide a lot of value.

    \vspace{0.5cm}

  \item[Syntax] A well-defined descriptive declarative semantics denotes mappings between versions of system state.

    Is change of relations a part of relationale algebra? This theorem highlights that whatever the denotations are they must be able to map states to accomodate the changes in syntax.

    Existing DDL appears to satisfy this theorem. Alter and drop definitely map system states.
\end{description}

\section{Syntactic Domains}
\begin{align*}
  A &: Alter \\
  C &: Column \\
  D &: Declaration \\
  Dr &: Drop \\
  I &: Identifier \\
  P &: Program \\
  Ta &: Table \\
  Ty &: Type
\end{align*}

\section{Production Rules}
\begin{grammar}
  <Program> := <Declaration>;

  <Declaration> := <Declaration>; <Declaration> | <Create> | <Alter> | <Drop>

  <Create> := \lit*{create table} <Identifier> ( <Column> )

  <Drop> := \lit*{drop table} <Identifier>

  <Column> := <Column>, <Column> | <Identifier> <Type>

  <Alter> := \lit*{alter table} <Identifier> \lit*{add} <Identifier> <Type>
           \alt \lit*{alter table} <Identifier> \lit*{drop} <Identifier>

  <Type> := \lit*{integer} | \lit*{boolean}
\end{grammar}

\section{Semantic Domains}
\begin{align*}
  Schema &: Identifier -> Table \\
  Tables &: Identifier -> Column \\
  Type &: Integer + Boolean
\end{align*}

\section{Semantic Functions}
\begin{align*}
  schema &: Schema \\
  schema &: \lambda i.\bot \\
  schemaChange &: Identifier -> Table -> Schema -> Schema \\
  schemaChange &= \lambda i. \lambda t. \lambda s. [ i \mapsto t ]s \\
  \\
  table &: Table \\
  table &: \lambda i.\bot \\
  tableChange &: Identifier -> Type -> Table -> Table \\
  tableChange &= \lambda i. \lambda ty. \lambda ta. [ i \mapsto ty ]ta
\end{align*}

\section{Valuation Functions}

\begin{alignat*}{2}
  &\mathcal{S} : Declaration -> Schema -> Sch&&ema \\
  &\cmp{S}{D;}s &&= \cmp{S}{D}s \\
  &\cmp{S}{D1;D2}s &&= \cmp{S}{D2}\ (\cmp{S}{D1}\ s) \\
  &\cmp{S}{create table I ( C )} s &&= schemaChange\ \mathtt{I}\ (\cmp{T}{C}\ table)\ s \\
  &\cmp{S}{drop table I}s &&= schemaChange\ \mathtt{I}\ \bot\ s \\
  &\cmp{S}{alter table I1 add I2 Ty}s &&= schemaChange\ \mathtt{I1}\ (\cmp{T}{\mathtt{I2} Ty}\ (s\ \mathtt{I1}))\ s\\
  &\cmp{S}{alter table I1 drop I2}s &&= schemaChange\ \mathtt{I1}\ (tableChange\ \mathtt{I2}\ \bot\ (\mathtt{s}\ \mathtt{I1}))\ s \\
\\
  &\mathcal{T} : Declaration -> Table -> Table && \\
  &\cmp{T}{C1, C2}t &&= \cmp{T}{C2}\ (\cmp{T}{C1}\ t) \\
  &\cmp{T}{I Ty}t &&= tableChange\ \mathtt{I}\ \mathtt{Ty}\ t \\
\end{alignat*}

\newpage

\section{Syntactic Domains}
\begin{align*}
  C &: Column \\
  D &: Declaration \\
  I &: Identifier \\
  P &: Program \\
  Ta &: Table \\
  Ty &: Type
\end{align*}

\section{Production Rules}
\begin{grammar}
  <Program> := <Declaration>;

  <Declaration> := <Declaration>; <Declaration> | <Create> | <Alter> | <Drop>

  <Create> := \lit*{create table} <Identifier> ( <Column> )

  <Column> := <Column>, <Column> | <Identifier> <Type>

  <Type> := \lit*{integer} | \lit*{boolean}
\end{grammar}

\section{Semantic Domains}
\begin{align*}
  Schema &: Identifier -> Table \\
  Tables &: Identifier -> Type \\
  Type &: Integer + Boolean
\end{align*}

\section{Semantic Functions}
\begin{align*}
  schema &: Schema \\
  schema &: \lambda i.\bot \\
  schemaChange &: Identifier -> Table -> Schema -> Schema \\
  schemaChange &= \lambda i. \lambda t. \lambda s. [ i \mapsto t ]s \\
  \\
  table &: Table \\
  table &: \lambda i.\bot \\
  tableChange &: Identifier -> Type -> Table -> Table \\
  tableChange &= \lambda i. \lambda ty. \lambda ta. [ i \mapsto ty ]ta
\end{align*}

\newpage

\section{Valuation Functions}

\begin{alignat*}{2}
  &\delta \in \{-, +, \eta \} &&\\
  &\mathcal{S} : Differential -> Schema -> &&Schema \\
  &\cmp{S}{D;}^{\delta}s &&= \cmp{T}{D}^{\delta}\ s \\
  &\cmp{S}{D1;D2}^{\delta}s &&= \cmp{T}{D2}^{\delta}\ (\cmp{T}{D1}^{\delta}\ s) \\
  &\cmp{S}{create table I ( C )}^{-}s &&= schemaChange\ \mathtt{I}\ \bot\ s\\
  &\cmp{S}{create table I ( C )}^{+}s &&= schemaChange\ \mathtt{I}\ (\cmp{T}{C}\ table) \ s \\
  &\cmp{S}{create table I ( C )}^{\eta}s &&= schemaChange\ \mathtt{I}\ (\cmp{T}{C}\ (s\ \mathtt{I})) \\
\\
  &\mathcal{T} : Differential -> Table ->&& Table\\
  &\cmp{T}{C1, C2}^{\delta}t &&= \cmp{T}{C2}^{\delta}\ (\cmp{T}{C1}^{\delta}\ t) \\
  &\cmp{T}{I Ty}^{-}t &&= tableChange\ \mathtt{I}\ \bot\ t\\
  &\cmp{T}{I Ty}^{+}t &&= tableChange\ \mathtt{I}\ \mathtt{Ty}\ t\\
  &\cmp{T}{I Ty}^{\eta}t &&= t \\
\end{alignat*}

\begin{description}
  \item Looking at this makes me wonder if all descriptive declarative languages boil down to tree manip. Maybe this is just a property of languages in general?

  \item \textbf{TODO} how to handle \lit*{create ; create} or \lit*{drop; drop} for the same identifier
\end{description}

\newpage

\section{Alternate Approach}
\begin{align*}
  schema &: Schema \\
  schema &= \emptyset \\
   \\
  relation &: Identifier -> Tuple -> Relation \\
  relation&\ I\ = I\emptyset
\end{align*}

\begin{alignat*}{2}
  &\delta \in \{-, +, \eta \} &&\\
  &\mathcal{S} : Differential -> Schema -> &&Schema \\
  &\cmp{S}{D;}^{\delta}S &&= \cmp{S}{D}^{\delta}\ S \\
  &\cmp{S}{D1;D2}^{\delta}S &&= \cmp{S}{D2}^{\delta}\ (\cmp{S}{D1}^{\delta}\ S) \\
  &\cmp{S}{create table I ( C )}^{-}S &&= S\, \backslash \langle r_{\mathtt{I}}\rangle \\
  &\cmp{S}{create table I ( C )}^{+}S &&= S^{\frown}\langle \cmp{R}{C}\ (relation\ \mathtt{I})\rangle \\
  &\cmp{S}{create table I ( C )}^{\eta}S &&= S\, \backslash \langle r_{\mathtt{I}}\rangle^{\frown} \langle \cmp{R}{C}^{\eta}\ r_{\mathtt{I}}\rangle \\
  \\
  &\mathcal{R} : Differential -> Relation -> && Relation \\
  &\cmp{R}{C1, C2}^{\delra}r &&= \cmp{R}{C2}^{\delta}\ (\cmp{R}{C1}^{\delta}\ r) \\
  &\cmp{R}{I Ry}^{-}r &&= r \backslash (\mathtt{I}) \\
  &\cmp{R}{I Ry}^{+}r &&= r^{\frown} (I) \\
  &\cmp{R}{I Ry}^{\era}r &&= r
\end{alignat*}

\newpage

\section{Problems Solved}

\begin{enumerate}
  \item \lit*{create X if exists ... ; create X if exists...} as a no-op
  \item \lit*{create} or \lit*{drop} on missing identifier
  \item failing midway through a DDL execution
  \item large table updates, e.g. adding two columns can be a create/swap
\end{enumerate}

\newpage

\bibliographystyle{plain}
\bibliography{ctm}

\end{document}
