\documentclass[12pt]{article}
\usepackage{bbold}
\usepackage{semantic}
\usepackage{mathtools}
\usepackage{syntax}
\usepackage{xstring}
\usepackage{trfrac}
\usepackage{listings}

\renewcommand{\syntleft}{\normalfont\itshape}
\renewcommand{\syntright}{}

\newcommand{\cmp}[2]{
  \mathcal{#1}|[ \mathtt{ \StrSubstitute{#2}{ }{\ } } |]
}

\newcommand{\syn}[1]{
  \synt{\textbf{#1}}
}

\newcommand{\ul}[1]{
  \underline{#1}
}

\begin{document}

\lstset{
  language=C,
  basicstyle=\ttfamily
}


\section{Abstract}

Roy and Haridi \cite{RoyH2004} divide declarative languages in to two major groups based on expressive power: descriptive and programmable. In particular, descriptive declarative languages are characterized by the their limited expressive power. Indeed, languages like SQL DDL that might otherwise be purely descriptive are augmented with programmable declarative elements and even their descriptive declarative elements are dependent on declaration order.

We believe that the purely descriptive form is sufficiently expressive in many cases given additional information about syntactic change that is readily available in modern programming environments. To leverage that information, we develop an extension to traditional denotational semantics that supports meaning for changes in syntax. With that, we construct a denotational semantics for standardized SQL DDL and show that it is possible to eliminate several of its non-descriptive features including statement order dependency. We examine the results of eliminating those features on several large database schema definitions over time and construct an adapter for a specific popular open source SQL implementation.

\begin{description}
  \item[eliminate unnecessary features] Show equivalence?
\end{description}


\section{Candidate Theorems}
\label{sec:theorems}

Order of preference:

\begin{description}
  \item[Syntax] A well-defined descriptive declarative syntax can always be mapped directly to the desired system state.

    This follows the same line of reasoning as the other syntax theorem: i.e., the declaration should dictate system state. The wording is focused on allowing snippets of syntax within a possibly larger grammar to satisfy the theorem.

    This sounds bi-directional, but that depends on the system. The syntax should determine the system state at interpretation time. It might be that other processes alter the state after that.

    Are things like the implementation of finding the difference between versions of the syntax unimportant for the theorem?

    Should be order independent, because part of the issue with reasoning about large DDL scripts is they are top to bottom order dependent. Not sure how to roll this in to the theorem.

    \vspace{0.5cm}

  \item[Syntax] A well-defined descriptive declarative semantics means there is always a clear mapping between syntax and system state.

    Note that \verb|alter| or \verb|drop| statements followed by more of the same for the same target object and properties will violate this theorem.

    This seems strict. Partial definitions may be more difficult to implement or prove but can still provide a lot of value.

    \vspace{0.5cm}

  \item[Syntax] A well-defined descriptive declarative semantics denotes mappings between versions of system state.

    Is change of relations a part of relationale algebra? This theorem highlights that whatever the denotations are they must be able to map states to accomodate the changes in syntax.

    Existing DDL appears to satisfy this theorem. Alter and drop definitely map system states.

    \vspace{0.5cm}

  \item[Syntax]


\end{description}

\section{Syntactic Domains}
\begin{align*}
  A &: Alter \\
  C &: Column \\
  D &: Declaration \\
  Dr &: Drop \\
  I &: Identifier \\
  P &: Program \\
  Ta &: Table \\
  Ty &: Type
\end{align*}

\section{Production Rules}
\begin{grammar}
  <Program> := <Declaration>;

  <Declaration> := <Declaration>; <Declaration> | <Create> | <Alter> | <Drop>

  <Create> := \lit*{create table} <Identifier> ( <Column> )

  <Drop> := \lit*{drop table} <Identifier>

  <Column> := <Column>, <Column> | <Identifier> <Type>

  <Alter> n:= \lit*{alter table} <Identifier> \lit*{add} <Identifier> <Type>
           \alt \lit*{alter table} <Identifier> \lit*{drop} <Identifier>

  <Type> := \lit*{integer} | \lit*{boolean}
\end{grammar}

\section{Semantic Domains}
\begin{align*}
  Schema &: Identifier -> Table \\
  Tables &: Identifier -> Column \\
  Type &: Integer + Boolean
\end{align*}

\section{Semantic Functions}
\begin{align*}
  schema &: Schema \\
  schema &: \lambda i.\bot \\
  schemaChange &: Identifier -> Table -> Schema -> Schema \\
  schemaChange &= \lambda i. \lambda t. \lambda s. [ i \mapsto t ]s \\
  \\
  table &: Table \\
  table &: \lambda i.\bot \\
  tableChange &: Identifier -> Type -> Table -> Table \\
  tableChange &= \lambda i. \lambda ty. \lambda ta. [ i \mapsto ty ]ta
\end{align*}

\section{Valuation Functions}

\begin{alignat*}{2}
  &\mathcal{S} : Declaration -> Schema -> Sch&&ema \\
  &\cmp{S}{D;}s &&= \cmp{S}{D}s \\
  &\cmp{S}{D1;D2}s &&= \cmp{S}{D2}\ (\cmp{S}{D1}\ s) \\
  &\cmp{S}{create table I ( C )} s &&= schemaChange\ \mathtt{I}\ (\cmp{T}{C}\ table)\ s \\
  &\cmp{S}{drop table I}s &&= schemaChange\ \mathtt{I}\ \bot\ s \\
  &\cmp{S}{alter table I1 add I2 Ty}s &&= schemaChange\ \mathtt{I1}\ (\cmp{T}{\mathtt{I2} Ty}\ (s\ \mathtt{I1}))\ s\\
  &\cmp{S}{alter table I1 drop I2}s &&= schemaChange\ \mathtt{I1}\ (tableChange\ \mathtt{I2}\ \bot\ (\mathtt{s}\ \mathtt{I1}))\ s \\
\\
  &\mathcal{T} : Declaration -> Table -> Table && \\
  &\cmp{T}{C1, C2}t &&= \cmp{T}{C2}\ (\cmp{T}{C1}\ t) \\
  &\cmp{T}{I Ty}t &&= tableChange\ \mathtt{I}\ \mathtt{Ty}\ t \\
\end{alignat*}

\newpage

\section{Syntactic Domains}
\begin{align*}
  C &: Column \\
  D &: Declaration \\
  I &: Identifier \\
  P &: Program \\
  Ta &: Table \\
  Ty &: Type
\end{align*}

\section{Production Rules}
\begin{grammar}
  <Program> := <Declaration>;

  <Declaration> := <Declaration>; <Declaration> | <Create> | <Alter> | <Drop>

  <Create> := \lit*{create table} <Identifier> ( <Column> )

  <Column> := <Column>, <Column> | <Identifier> <Type>

  <Type> := \lit*{integer} | \lit*{boolean}
\end{grammar}

\section{Semantic Domains}
\begin{align*}
  Schema &: Identifier -> Table \\
  Tables &: Identifier -> Type \\
  Type &: Integer + Boolean
\end{align*}

\section{Semantic Functions}
\begin{align*}
  schema &: Schema \\
  schema &: \lambda i.\bot \\
  schemaChange &: Identifier -> Table -> Schema -> Schema \\
  schemaChange &= \lambda i. \lambda t. \lambda s. [ i \mapsto t ]s \\
  \\
  table &: Table \\
  table &: \lambda i.\bot \\
  tableChange &: Identifier -> Type -> Table -> Table \\
  tableChange &= \lambda i. \lambda ty. \lambda ta. [ i \mapsto ty ]ta
\end{align*}

\newpage

\section{Valuation Functions}

\begin{alignat*}{2}
  &\delta \in \{-, +, \eta \} &&\\
  &\mathcal{S} : Differential -> Schema -> &&Schema \\
  &\cmp{S}{D;}^{\delta}s &&= \cmp{T}{D}^{\delta}\ s \\
  &\cmp{S}{D1;D2}^{\delta}s &&= \cmp{T}{D2}^{\delta}\ (\cmp{T}{D1}^{\delta}\ s) \\
  &\cmp{S}{create table I ( C )}^{-}s &&= schemaChange\ \mathtt{I}\ \bot\ s\\
  &\cmp{S}{create table I ( C )}^{+}s &&= schemaChange\ \mathtt{I}\ (\cmp{T}{C}\ table) \ s \\
  &\cmp{S}{create table I ( C )}^{\eta}s &&= schemaChange\ \mathtt{I}\ (\cmp{T}{C}\ (s\ \mathtt{I})) \\
\\
  &\mathcal{T} : Differential -> Table ->&& Table\\
  &\cmp{T}{C1, C2}^{\delta}t &&= \cmp{T}{C2}^{\delta}\ (\cmp{T}{C1}^{\delta}\ t) \\
  &\cmp{T}{I Ty}^{-}t &&= tableChange\ \mathtt{I}\ \bot\ t\\
  &\cmp{T}{I Ty}^{+}t &&= tableChange\ \mathtt{I}\ \mathtt{Ty}\ t\\
  &\cmp{T}{I Ty}^{\eta}t &&= t \\
\end{alignat*}

\begin{description}
  \item Looking at this makes me wonder if all descriptive declarative languages boil down to tree manip. Maybe this is just a property of languages in general?

  \item \textbf{TODO} how to handle \lit*{create ; create} or \lit*{drop; drop} for the same identifier
\end{description}

\newpage

\section{Alternate Approach}
\begin{align*}
  schema &: Schema \\
  schema &= \emptyset \\
   \\
  relation &: Identifier -> Tuple -> Relation \\
  relation&\ I\ = I\emptyset
\end{align*}

\begin{alignat*}{2}
  &\delta \in \{-, +, \eta \} &&\\
  &\mathcal{S} : Differential -> Schema -> &&Schema \\
  &\cmp{S}{D;}^{\delta}S &&= \cmp{S}{D}^{\delta}\ S \\
  &\cmp{S}{D1;D2}^{\delta}S &&= \cmp{S}{D2}^{\delta}\ (\cmp{S}{D1}^{\delta}\ S) \\
  &\cmp{S}{create table I ( C )}^{-}S &&= S\, \backslash \langle r_{\mathtt{I}}\rangle \\
  &\cmp{S}{create table I ( C )}^{+}S &&= S^{\frown}\langle \cmp{R}{C}\ (relation\ \mathtt{I})\rangle \\
  &\cmp{S}{create table I ( C )}^{\eta}S &&= S\, \backslash \langle r_{\mathtt{I}}\rangle^{\frown} \langle \cmp{R}{C}^{\eta}\ r_{\mathtt{I}}\rangle \\
  \\
  &\mathcal{R} : Differential -> Relation -> && Relation \\
  &\cmp{R}{C1, C2}^{\delra}r &&= \cmp{R}{C2}^{\delta}\ (\cmp{R}{C1}^{\delta}\ r) \\
  &\cmp{R}{I Ry}^{-}r &&= r \backslash (\mathtt{I}) \\
  &\cmp{R}{I Ry}^{+}r &&= r^{\frown} (I) \\
  &\cmp{R}{I Ry}^{\era}r &&= r
\end{alignat*}

\newpage

\section{Problems Solved}

\begin{enumerate}
  \item \lit*{create X if exists ... ; create X if exists...} as a no-op
  \item \lit*{create} or \lit*{drop} on existing/missing identifier
  \item failing midway through a DDL execution
  \item large table updates, e.g. adding two columns can be a create/swap
  \item analysis becomes easier. Might otherwise require simulation
  \item analysis can determine type conversion issues http://www.postgresql.org/docs/9.3/static/ddl-alter.html#AEN2787
  \item analysis can catch integrity constraint issues/failures?
  \item triplicate of all DDL
  \item detect change of column name order without change of selected columns in create table as
\end{enumerate}

\section{Notes}

\subsection{General}

\begin{enumerate}
  \item Schema evolution issues are most often handled after the new schema is defined. That is tools like PRISM analyze the new schema and then determine how to get there with the data/queries etc. If the newly defined schema is flawed in the first place these tools are of little value.
  \item Comparison of schemas from large and complex databases before and after with differential semantics.
  \item Schema history is trivial to track using source control.
  \item There appears to be two parts to this, static analysis and differential semantics. They facilitate each other, order dependency graph is static analysis and the reduced syntax comes from differential semantics
  \item is there some way to characterize how the changes in the declarative syntax eliminate the other parts of the language more generally? How do you classify that as an addition of expressive power?
  \item take no action unless all updates will succeed -> well defined declarative syntax does not go wrong? Handled by transactions in some DBs (postgres), still requires system resources (possibly a lot in some cases) before rollback if failure comes late)
\end{enumerate}

\subsection{Possible Issues}

\begin{enumerate}
  \item line level merge conflicts can be deceptive/destructive, not a problem with sequential alters and drops
  \item do schema qualifier changes constitute a move or a copy?
  \item \lit*{create table as ...} changes may be data destructive
  \item \lit*{create table as ...} drop and recreate? remove data that doesn't match?
  \item \lit*{create table as ...} defining the semantics for data removal requires query inversion
  \item order matters \lit*{create table foo ...; create table as select * from foo ...} will work \lit*{create table as select * from foo; create table foo ...; } won't
  \item foreign key constraints are order and data dependent (DML has to be right when adding constraints)
  \item triggers might prevent DDL updates
\end{enumerate}

\subsection{Order/State}

\begin{enumerate}
  \item order should be unimportant in a declarative language
  \item any expectations about the existing state should be handled (eg foreign key constraints require certain data)
\end{enumerate}

\subsection{Static Analysis}
\begin{enumerate}
  \item dml insertions into table must respect data constraints (eg, not null, types)
  \item historical information constraints (eg, if a column is/has always been not null the data will always satisfy the constraint)
  \item build a set of constraints for the tables, check DML statments (data later still has value even if changing)
  \item identity dependencies
  \item index table dependencies
  \item referential constraint dependencies
  \item column constraint dependencies (foo int check foo > bar ==> foo depends on bar)
  \item as column constraints change over time they might become incompatible
  \item constraint solving over time, temporal understanding of constraints
  \item certain types of constraints suggest certain types of index bitmap vs normal (eg, foo int CHECK (price > 0 and price < 10))
  \item constraints could allow for fast verification of data (eg, foo int CHECK (price > 0 and price =< 10 + unique) ==> check the table for 10 items) on foreign key checks
  \item expansion (weakening?) of constraints is sure to be ok, strengthening may not be
  \item redundant constraints
  \item pk => unique + not null, many columns is more complex
  \item delete restrictions/cascades vs other constraints?
  \item constraint relationship with queries for redundant, unnecessary, or impossible queries
  \item \lit*{insert into select ...} and \lit*{create table as ... } will satisfy the constraints of the table being selected from, even if the table doesn't require them
  \item \lit*{insert into select ...} data set should satisfy constraints of table definition
  \item \lit*{create index ...} predicates (postgres) should satisfy constraints
  \item constraint solving can produce different answers for the existing data vs the new schema, (eg, constraint removed in current schema, still constrained on existing data, reliance on existing data must reference old constraint, reliance on new schema must reference new constraint)
  \item type mismatch in definition and use of columns
  \item type mismatch in alteration of columns
  \item type width as a subtype relationship?
  \item what is the relationship between view/create as where statement constraints and regular table constraints?
  \item are constraints other than not null / default used?
\end{enumerate}

\subsection{Constraints/Types}

\subsubsection{Type definitions}

\begin{enumerate}
  \item columns and their types are type definitions
  \item column/table constraints are dependent type constraints
  \item referential constraints are dependent type constraints from external sources
  \item constraints are a weak form a schema modeling
\end{enumerate}

\subsubsection{Type checks}

\begin{enumerate}
  \item columns references must respect type constraints
  \item where/having/order clauses should respect type/data constraints
  \item join results should respect proxied referential data constraints
  \item column type alterations must/should respect type constraints
\end{enumerate}

\subsubsection{Value}
\begin{enumerate}
  \item Prevent type mismatch
    \begin{enumerate}
      \item selection, \lit*{... LIKE ``foo'' ...} on \lit*{int} column
      \item projection, \lit*{ ... lower(column_name) ... } on \lit*{int} column
      \item alteration, \lit*{ ALTER TABLE ... TYPE numeric(10,2);} on \lit*{text} column
    \end{enumerate}
  \item Prevent uselessness
    \begin{enumerate}
      \item e.g. selection criteria fall outside table constraints, especially inserts/creates
    \end{enumerate}
\end{enumerate}

\subsubsection{Questions}
\begin{enumerate}
  \item How do you concat two types in a join? See paper from Jens?
\end{enumerate}

\newpage

\section{Transformations}

\begin{enumerate}
  \item Questions
    \begin{enumerate}
      \item How can you ``remember'' computation or delay it and then ``cleanup''
      \item Does applying them one at a time change anything?
      \item Is there work on program transformation that applies?
      \item Do monadic computations generally provide checkpoints? Monadic ``loops''?
      \item Instruction reordering an issue for updates?
      \item Is Haskell's \texttt{runState} an \texttt{update} equivalent?
      \item Where does the diff begin end?
      \item Could side effects force the computation?
      \item How can you safely do partial updates?
      \item What can be done without types?
    \end{enumerate}

  \item Notes
    \begin{enumerate}
      \item alpha equivalence is a no-op
      \item Combining ``in-flight'' updates with checkpoint has value when the developer misplaces/forgets the \texttt{update} or the execution takes to long to reach it.
      \item Replacing without the update/timeout requirement removes cognitive overhead.
      \item Prove: handles modules (through let?)
      \item Prove: same update never applied twice
      \item Prove: When updates are added to $\Delta$ they always take place before the term is reduced
      \item Prove: updates only apply to original terms
      \item Should preserve the language semantics
      \item A change requires type checking in old and new environment
    \end{enumerate}

  \item TODO
    \begin{enumerate}
      \item Beta equivalence (undecidable)
      \item $\Delta$ ``cleanup'' for ``unused'' updates
      \item Type checking
      \item Reduced/Partially Applied term updates
      \item Working on impure functions
    \end{enumerate}

\newpage

  \item Pre-flight Updates

    Simple

    \begin{equation*}
      (\lambda z.\lambda x.\lambda y.x)^0\ (\lambda x.x)^{10}
    \end{equation*}

    \begin{equation*}
      \Delta = \{(\lambda x.\lambda y.x,\
      \ul{\lambda n}.\lambda x.\lambda y.x\ \ul{n})^{3} \}
    \end{equation*}

    \begin{equation*}
      (\lambda z.\lambda x.\lambda y.x)^0\ (\lambda x.x)^{10} \rightarrow
      (\lambda x.\lambda y.x)^3 \rightarrow
      (\ul{\lambda n}.\lambda x.\lambda y.x\ \ul{n})^3
    \end{equation*}


    \begin{equation*}
      \trfrac[E-UpdateLambda]{
        (\lambda x.t_1, t)^x \in \Delta \qquad \Delta' = \Delta \setminus (\lambda x.t_1, t)^x
      }{
        (\lambda x.t_1)^x\ t_2 ->
        t\ t_2\
      }
    \end{equation*}

    \begin{equation*}
      \trfrac[E-UpdateBody]{
        (t_1, t)^{x} \in \Delta \qquad \Delta' = \Delta \setminus (t_1, t)^{x}
      }{
        (\lambda x.t_1)^{x}\ t_2 ->
        (\lambda x.t)^{x}\ t_2
      }
    \end{equation*}

    \begin{equation*}
      \trfrac[E-AppBeta]{
        (\lambda x.t_1,\_), (t_1, \_) \notin \Delta
      }{
        (\lambda x.t_1)^{x}\ t_2 ->
        [x \mapsto t_2]\,t_1^{\;x+3 }
      }
    \end{equation*}

    \begin{equation*}
      \trfrac[E-Update]{
        \Delta -> \Delta \cup \Delta'
      }{
        t -> t
      }
    \end{equation*}

  \item Explicit Updates

    \begin{equation*}
      \trfrac[E-Update]{
        \Delta -> \Delta \cup \Delta'
      }{
        \mathtt{update}\ t -> t
      }
    \end{equation*}

  \item Updates for Reduced Terms

    \begin{equation*}
      \trfrac[E-AppBetaPart]{
        dom(\langle \ldots \rangle)(\lambda x.\lambda y.t_1,\_) \notin \Delta
      }{
        \langle \ldots \rangle(\lambda x.\lambda y.t_1)^{x}\ t_2 ->
        \langle \ldots, x \mapsto t_2 \rangle(\lambda y.t_1)^{\;x+3 }
      }
    \end{equation*}

    \begin{equation*}
      \trfrac[E-AppBetaSubApp]{
        (\lambda x.t_1\ t_2,\_) \notin \Delta
      }{
         \langle \ldots \rangle (\lambda x.t_1\ t_2)^{x}\ t ->
         [ \ldots, x \mapsto t ] \,(t_1\ t_2)^{\;x+3 }
      }
    \end{equation*}

    \begin{equation*}
      \trfrac[E-AppBeta]{
        (\lambda x.var,\_) \notin \Delta
      }{
         \langle \ldots \rangle (\lambda x.var)^{x}\ t ->
         [ \ldots, x \mapsto t] var^{\;x+3 }
      }
    \end{equation*}

  \item Pure update example
    \lstinputlisting[language=C]{code/rewind.c}

\end{enumerate}

\newpage

\bibliographystyle{plain}
\bibliography{ctm}

\end{document}
